%%%%%%%%%%%%%%%%%%%%%%%%%%%%%%%%%%%%%%%%%%%%%%%%%%%%%%%%%%%%%%%%%%%%%%%%%%%%%%%%%%%%%%%%%%%%%
% 																STATE OF THE ART 																					%
%%%%%%%%%%%%%%%%%%%%%%%%%%%%%%%%%%%%%%%%%%%%%%%%%%%%%%%%%%%%%%%%%%%%%%%%%%%%%%%%%%%%%%%%%%%%%
\chapter{Related Work}


\subsection{User Preferences}
\label{sub:user preference}
Researchers have tried to figure out the challenges which will be involved in
placing robots in a human environment.\cite{pantofaru_exploring_2012} conducts
interviews with people to understand the needs of people with robots. One of the
suggestion was help in organizing things based on the user preferences.
For understanding user preferences in organizing objects in home work has 
been done by  \cite{abdo2015robot}, where robots learn about user
preferences in organizing objects. The learning uses data collected by 
crowdsourced data collected from thousands of users to predict location of 
novel objects. Our work differs from above as we only use data collected from a single home and predict object locations for the same home. Thus the learning is to understand and reason about a single home and not about a generalized home.

\subsection{Knowledge acquisition in mobile robots}
Current state of the art knowledge reasoning systems such as KnowRob \cite{tenorth2013knowrob} use publicly available knowledge bases like Cyc \cite{lenat1995cyc} and Open-Mind Indoor Common Sense (OMICS) database \cite{singh2002open} for gaining base knowledge about environments.  Although these methods capture generic information of human environments they lack in capturing the knowledge about single user preferences and habits. \cite{niemueller2012generic} proposes the use of document-oriented, schema-less database as robot memory, and using the database to generate knowledge. \cite{mason2012object} also proposes on same front where the robot continuously records its observations and then queries to find knowledge about change in the environments.


\subsection{Knowledge enabled search of objects}
\label{sub:partially known objects}
Object contextual information for searching objects as been repeatedly proven good results. Many search strategies involving searching based on known object locations have been developed
\cite{kollar_utilizing_2009} utilize object-object and object-place co-occurrences probabilities as a way to shape the prior on the object location over the search space. Using prior map of the environment and knowledge about some of the objects in it, they try to search for location of novel objects.
Extension to this approach involving information retrieval from the web has been explored by \cite {samadi_using_2012}
\cite{kunze_searching_2012} applied the semantic similarity measure for object search by using prior information from a web-trained ontology.
Finally, \cite{joho_learning_2011} illustrated one way of combining both types of information by extracting features to train a reactive search heuristics.

\cite{wong_using_2014} considered the case of occluded objects. Occluding objects in the front typically need to be moved away to enable further perception and eventual discovery of such occluded objects. 
% subsection partially known objects (end)


