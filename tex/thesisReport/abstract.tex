%%%%%%%%%%%%%%%%%%%%%%%%%%%%%%%%%%%%%%%%%%%%%%%%%%%%%%%%%%%%%%%%%%% 
%                                                                 %
%                            ABSTRACT                             %
%                                                                 %
%%%%%%%%%%%%%%%%%%%%%%%%%%%%%%%%%%%%%%%%%%%%%%%%%%%%%%%%%%%%%%%%%%% 
\begin{abstract}


Domestic service robots are envisioned to provide services to humans in both home and office environments. Examples include domestic service robots, that implements large part of our housework, versatile assistants, that provide automation, transportation inspection and monitoring services. The challenge in these applications is that the robots need to have lot of information and knowledge about the user and environment to operate under complete autonomy. This thesis aim is to enable domestic robots to acquire such knowledge about behaviour and preferences of the users around them. The developed approaches in this thesis cover the following two topics: (1) learning about the temporal human location behaviour in-door using previously observed locations of the human (2) learning user preference in object placement. 
 
All techniques developed in this thesis are based on probabilistic Bayesian learning and inference. They have been implemented and evaluated on datasets  collected using real robots as well as on simulated datasets. Extensive experiments have been conducted to evaluate the and validate the properties of the proposed models.
\todo{TODO : need to be add results and revise}
\end{abstract}