%%%%%%%%%%%%%%%%%%%%%%%%%%%%%%%%%%%%%%%%%%%%%%%%%%%%%%%%%%%%%%%%%%% 
%                                                                 %
%                            ABSTRACT                             %
%                                                                 %
%%%%%%%%%%%%%%%%%%%%%%%%%%%%%%%%%%%%%%%%%%%%%%%%%%%%%%%%%%%%%%%%%%% 
\begin{abstract}

The goal of this thesis is to enable domestic service robots to gain knowledge about common behaviours and preferences of the user in a non-intrusive manner. Specifically, we address how Bayesian methods can be used to provide a flexible and computationally efficient structure for acquiring knowledge using limited spatio-temporal information collected by these service robots.

Domestic service robots are envisioned to provide assistance to humans both in domestic environments as well as in the industrial context. One of the ways domestic service robots can better serve humans in assisting, is by providing personalized, predictive and context-aware services. Robots can learn observing human activities and work-life patterns of humans and provide predictions that enable time based contextual assistance.  This thesis aims to enable domestic robots to empirically learn about human behaviour and preferences. 
Robots generate a lot of information using the raw data from its sensors, which is often discarded after use. If these informations are recorded it can be used to generate new knowledge. The thesis aims to generate knowledge about the human preferences and behaviour using these stored informations. The developed approaches in this thesis cover the following two knowledge generation topics: (1) learning about the temporal human location behaviour using previously observed locations (2) learning users preference in object placement. 
All knowledge generation techniques developed in this thesis are based on Bayesian modelling and been implemented using probabilistic programming languages. They have been evaluated on datasets  collected using real robots as well as on synthetic datasets.
\todo[inline]{TODO : need to be add results and revise}
\todo[inline]{All the models were developed using probabilistic programming languages }
\end{abstract}