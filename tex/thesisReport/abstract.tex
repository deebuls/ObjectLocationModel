%%%%%%%%%%%%%%%%%%%%%%%%%%%%%%%%%%%%%%%%%%%%%%%%%%%%%%%%%%%%%%%%%%% 
%                                                                 %
%                            ABSTRACT                             %
%                                                                 %
%%%%%%%%%%%%%%%%%%%%%%%%%%%%%%%%%%%%%%%%%%%%%%%%%%%%%%%%%%%%%%%%%%% 
\begin{abstract}


One of the ways domestic service robots can better assist humans is by providing personalized, predictive and context-aware services. Robots can observe human activities and work patterns to provide time-based contextual assistance. This thesis aims to enable domestic robots to empirically learn about human behaviour and preferences. In the current literature, user preferences are learned over a generic home environment, on the contrary we learn preferences over a specific home.
Robots generate a lot of information using the raw data from their sensors, which is often discarded after use. If this information is recorded, it can be used to generate new knowledge. The thesis proposes models to generate knowledge about user preferences using stored information. The developed approaches in this thesis cover the following two knowledge generation topics: (1) learning user location preferences (2) learning user preferences in object placement. 

All knowledge generation techniques developed in this thesis are based on Bayesian modelling and have been implemented using probabilistic programming languages
The learned user preferences were used by the robot for predicting: (1) location of non-stationary objects (2) location of users in home and (3) room occupancy.  
The models were evaluated on three datasets collected over several months containing person and object occurrences in home and office environments. 
The efficiency of the models was accessed by measuring the accuracy score of each models. Our model for predicting location of non-stationary objects was able to predict with 70\% accuracy for 25 objects, while model for room occupancy could predict for 3 rooms with more than 80\% accuracy. The model for user location preference showed a poor performance of 63\%.

\end{abstract}