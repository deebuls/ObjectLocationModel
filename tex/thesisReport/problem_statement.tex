
\chapter{Problem Formulation}
\label{sec:Problem formulation}

The central ideas of the thesis is to use Bayesian models for representing the learned knowledge about human behaviour and preferences. We formulate the problem as getting an accurate probability density of possible locations given the previous observed locations. Given the corresponding observations of $D_{o_i}$, the probability distribution over the locations of $o_i$ at time $T$ is governed by the following formula 

    \begin{equation} \label{eq:1}
	    P(l_i | t_i, D_{o_i})
    \end{equation}

   Various temporal information related to periodic patterns can be implied by $T$, to indicate the location distribution. Such as specific hours of the day (11:00 pm), a day of the week(Friday), or a month of the year(February). We use the \textbf{temporal state} to represent such information and introduce $r(t)$ to denote temporal state extracted from time $T$,.  Dependency on the type of the temporal state $r(t)$ can be a different function. For example,if $r(t)$ denotes temporal state in terms of hours of the day then $r(t) \in {0,1 ... , 23}$, if $r(t)$ denotes temporal state in terms of day of the week, then $r(t) \in {0,1, .. 6}$ . Without loss of generality, we use $r(t)$ to denote a type of temporal state in the following description, Equation \ref{eq:1} is reformulated as 
    
    \begin{equation}
	    P( l_i | r(t), D_{o_i})
    \end{equation}
    
    Applying Bayes Rule
    
    \begin{equation}\label{eq:3}
	P( l_i | r(t), D_{o_i}) \propto P(r(t) | l_i, D_{o_i})  P(l_i | D_{o_i})
    \end{equation}
    Where:
    \begin{itemize}[label=]
    \item $P(r(t) | l_i, D_{o_i})$ : Temporal context 
    \item $P(l_i | D_{o_i})$ : Spatial context
    \end{itemize}
    
     The spatial context $P(l_i | D_{o_i})$ indicates the location distribution of object or person $o_i$ given the previous observed location $D_{o_i}$ . The temporal context $P(r(t) | l_i, D_{o_i})$ represents the temporal state distribution of object $o_i$, being observed at location $l_i$ with corresponding $D_{o_i}$
    


\begin{tabular}{cp{8cm}}
    \hline
	Symbol & Meaning\\
	\hline
	O & Set of all objects or persons\\
	$o_i$ & Single object from the set $O$, $o_i \in O$ \\
	$L$ & Set of all locations\\
	$l_i$ & Single location from the set $L$. $l_i\in L$\\
    $T$ & Time interval\\
    \hline
	$<o_i,l_i,t_i>$ & object $o_i$ was located at location $l_i$ at time $t_i$\\
	$D$ & Collection of all objects all observed locations\\
	$D_{o_i}$ & Previous observed locations of $o_i$\\ 
    \hline
     $P(r(t) | l_i, D_{o_i})$ &  Temporal context representing the temporal state distribution of $o_i$ at location $l_i$ given previous observations $D_i$\\
     $P(l_i | D_{o_i})$ & Spatial context representing the location distribution of object $o_i$ given the previous observations $D_{o_i}$\\
    \hline
\end{tabular}
% section Problem formulati (end)

\todo[inline]{
The central idea of the thesis is based on the strong assumption that humans do their activities in the context of time. Humans follow a certain routine in their lifestyles and time is one de-facto latent element in determination of these routines. Most of the activities in ones daily life can be accredited as being caused because the time of the day. or can be accredited to the time of the day ? . Thus time has a prominent place in the models of human behaviours and preferences. All the models developed in this thesis try to learn human behaviours and preferences based on time. 


Time has a prominent place in developing  model.

}
% section  Things to study (end)