%%%%%%%%%%%%%%%%%%%%%%%%%%%%%%%%%%%%%%%%%%%%%%%%%%%%%%%%%%%%%%%%%%%%%%%%%%%%%%%%%%%%%%%%%%%%%
%											IMPLEMENTATION AND MEASUREMENTS																				%
%%%%%%%%%%%%%%%%%%%%%%%%%%%%%%%%%%%%%%%%%%%%%%%%%%%%%%%%%%%%%%%%%%%%%%%%%%%%%%%%%%%%%%%%%%%%%
\chapter{EVALUATION}


For evaluation the proposed models 3 different approaches 
\begin{itemize}
	\item Artificial data generation \\
	A data generator will be written to generate the artificial dataset to simulate object location of different objects with respect to time.
	\item Human presence detection \\
	The ‘Aruba’ dataset by \cite{cook2010learning} contains measurements collected by
50 different sensors distributed over a 12×10 m, seven-
room apartment over a period of 16 weeks. The apartment is
occupied by a single person who is occasionally visited by
other people. The dataset will be used to estimate person presence in a particular room.

	\item Object detection \\
	The KTH dataset by \cite {krajnik_life-long_2015} was collected
by a SCITOS-G5 mobile robot, in the Computer Vision and Active Perception lab at KTH Stockholm,
over the course of five weeks. During this time the robot
conducted between two and six autonomous patrol runs
per day (weekends were excluded), visiting three specific
waypoints during each run. Upon reaching a waypoint, the
robot would execute a pan-tilt sweep and collect data from
its RGB-D sensor; the RGB-D frames collected during one
sweep were then registered spatially to form an observation
of that particular waypoint at that time. The KTH dataset
contains approximately 100 observations per waypoint, and
at each waypoint we extracted the dynamic elements of
the environment using the ‘MetaRoom’ method described
in \cite{}. These dynamic elements correspond to movable objects
such as jackets, backpacks, laptops, chairs, bottles, mugs,
etc. The objects were manually
labeled to these dynamic clusters to obtain 37 different objects,
out of which 14 tend to appear and disappear periodically.

\end{itemize}
