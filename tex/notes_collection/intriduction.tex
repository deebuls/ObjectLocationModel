\documentclass[11pt]{book}
%Gummi|063|=)
\title{\textbf{Complete this book }}
\author{Deebul Nair}
\date{}

\usepackage{amsmath}
\usepackage{todonotes}
\begin{document}



\chapter{Introduction}


In this thesis, we consider methods for learning dynamical models for time series with complex and uncertain behaviour patterns. Specifically, we address how Bayesian non-parametric methods can be used to provide a flexible and computationally efficient structure for learning and inference of these complex systems.

Some of the deepest questions of cognitive development are: How does abstract knowledge influence learning of specific knowledge? How can abstract knowledge be learned? In this section we will see how such hierarchical knowledge can be modeled with hierarchical generative models: generative models with uncertainty at several levels, where lower levels depend on choices at higher levels.

\begin{description}
	\item[asdf] 
\end{description}
\section{ The Right Kind of Smarts }
Smart robots humbly predict our needs and modestly adjust as little as possible to accommodate them. Imagine if your robot could learn how you arrange the breakfast table by looking at the data from previous days? 
We can also have a conversation with smart robots. They can tell us what they’re up to when we ask, or tell us something’s wrong when it’s essential. They can observe our lives and provide small insights we don’t even notice. They can pass along helpful information to humans, like observing our sleep habits and tell us when we are not having adequate sleeps.
We can have a new relationship with our robots, one where the previously mute boxes of plastic and metal become new platforms—not as replacement, but for meaning and value. By learning how we interact with our homes and how we live our lives, robots will be able to provide services to us we can’t see right now. They’ll set themselves up and fit into the existing household by knowing what—and who—is there and adapting to them. Robots will grow and change with you and the house robots with a greater awareness of the world around them.



\end{document}