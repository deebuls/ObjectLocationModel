\documentclass[11pt]{report}
%Gummi|063|=)
\title{\textbf{Complete this book }}
\author{Deebul Nair}
\date{}

\usepackage{amsmath}
\usepackage{todonotes}
\begin{document}

\maketitle


\chapter{Basics}

In this background chapter, we review the statistical and probabilistic methodologies .
The goal of this chapter is to provide the basic information regarding Bayesian methods of machine learning used in the thesis. A good introduction to the field can be found in books of Bishop \cite{}, Kirchop \cite{} and LeeWagenmakers \cite{} . 

\section{Bayesian Model Learning}

Bayesian models are represented as probability distributions. Probability is used to quantify "uncertainty" or "Degree of belief". The models are initialized with some prior probabilities(beliefs). The observed data are used to update the prior beliefs to become posterior beliefs.

We can explain this with an example. Assume we have a bag with 50 marbles with 2 colours, red and blue. We randomly take 10 marbles out of the bag and observe their colour with replacement. What we have to model is, our belief of the numbers of colours inside the bag, in bayesian  terms colour distribution in the bag, which we can define as $\theta$ . We cannot directly observe the bag. All that we can observe are the colours of the 10 picked marbles.

Before we do anything else we need to specify our prior belief with respect to the colour distribution $\theta$ . This belief needs to be expressed as a probability distribution, called the \emph{prior distribution}. A reasonable "prior distribution" denoted by $p(\theta)$ is one which takes uniform value between 0 and 1. Lets assume $p(\theta)$ is belief of red marbles in bag then $1 - p(\theta)$  is the belief of blue marbles in the bag. This uniform distribution is shown as dotted line in the figure 

Now we consider the observed colour of the 10 marbles from the bag. We observe 8 red marbles and 2 blue marbles. After observing the data, the updated knowledge about $\theta$ is described by a \emph{posterior distribution}, denoted by $p(\theta | D)$, where D indicates the observed data. The distribution represents the updated belief after observing the colour picked marbles. Bayes rule specifies how we can combine the information from the data, that is how to determine the posterior distribution $p (\theta | D)$ using  the prior distribution $p(\theta)$ and the likelihood  $p (D | \theta)$ :
\begin{equation}
	p(\theta | D) = \frac{p(D | \theta) p(\theta)}{p(D)}
\end{equation}

The equation is often verbalized as :
\begin{equation}
	posterior = \frac{likelihood * prior}{marginal likelihood}
\end{equation}

We note here that the posterior distribution is a combination of the prior information we had and what we have learned from the data. 



\missingfigure{Image of the prior posterior}

\section{Prediction}

\section{Distributions }

Dirichlet, Categorical, Bernoulli 
\subsection*{Dirichlet distribution}
The Dirichlet distribution is part of the exponential family. It has finite dimensional sufficient statistics. It is conjugate to the multinomial and categorical distribution. 

Conjugate prior  need to explain this 

\section{Graphical Models}

\section{Probabilistic Programming}

Until recently, Bayesian Model learning have been limited in scope, and have been hard to apply to many real-world applications. Probabilistic programming is a new approach which makes Bayesian learning easier to build and more applicable. 
\subsection{PyMC3}

\subsection{BayesPy}

\subsection{WebPPL}

\section{Model Evaluation}

\section{DIKW Pyramid }

\section{Notation and terminology}
Throughout the thesis, we are referring to entities such as ``locations," ``hours," and ``observations".
This helps to guide intuition and maintain continuity of thoughts as we guide through related problems involving collections of data.
Formally we define the following terms:
\begin{itemize}
	\item A \emph{location} is the basic unit of the discrete data, defined to be an item from a set of locations. These locations can be rooms of the home or different compartments of the kitchen. 
	\item An \emph{period} is a sequence of $N$ locations denoted by $\textbf{p} = {x_1;x_2;:::;x_N}$. These represents the locations observed in a particular period of time.
	\item A \emph{observations} is a collection of $T$ periods denoted by $ D = {p_1;p_2;:::;p_T}$. These represents the complete data collected by the robot.
\end{itemize}

\section{TO explain or to Predict}

From either a frequentist or a Bayesian perspective, it is essential to distinguish the
ultimate goal of modeling when confronting a statistical data analysis project. Geisser
and Eddy (1979) challenge research workers two fundamental questions that should
be asked in advance of any procedure conducted for model selection:
• Which of the models best explains a given set of data?
• Which of the models yields the best predictions for future observations from
the same process which generated the given set of data?
The former, which cares about how accurately a model describes the current data in
the explanatory point of view, has been the problem of empirical science for many
years; whereas the latter, which focuses on predicting future data as accurately as
possible in the predictive perspective, is more crucial and difficult to answer and has
drawn more attentions in recent decade.
If an infinitely large quantity of data is available, the predictive perspective and
the explanatory perspective might not differ significantly. With only limited number
of observations in practice, it is a more difficult task for predictive model selection

from " Bayesian Model Selection in terms of Kullback-Leibler discrepancy
Shouhao Zhou
%######################################################################

%\documentclass[11pt]{book}
%Gummi|063|=)
\title{\textbf{Complete this book }}
\author{Deebul Nair}
\date{}

\usepackage{amsmath}
\usepackage{todonotes}

\usepackage{tikz}
\usepackage{amsmath}
\usetikzlibrary{bayesnet}
\usepackage{tabularx}

\begin{document}

\chapter{In-House Human Presence Model}

Domestic robots in future should be able to gather knowledge about the favourite places of the humans in the home and also the time period when they mostly occupy their favourite places. This acquired knowledge of the human presence patterns shall enable the robots to make better decisions. For example what time particular rooms are unoccupied for cleaning or when to turn on the heater of the room. 

In this chapter we focus on such knowledge accession from observations made by the robot. However, the advancement in long term autonomous navigation \todo{cite} and the rapid adoption of databases in the robots \todo{cite} has not only made it feasible for domestic robots to acquire such knowledge, but also a number of challenges.  In more details, these challenges are the (i) modelling human presence (ii) prediction of future location (iii) learning all these with sparse observations made by the robots. 

Significant progress has been made on a related problem by researchers in the field of human location behaviour, which is to learn the patterns in human location outdoors. Approaches for learning routine mobility  range from purely temporal (\cite{c1, c2}), spatial (\cite{c5,c3}), to a combination  of  both  (\cite{c4}). Non-parametric Bayesian methods are also gaining popularity given their ability to refine models as more data arrives. Chen et al. (2012) used a Gaussian process to model congestion on road networks, while Gao et al. (2012) used a hierarchical Pitman Yor process to model check-in behaviour on location-based social networks. Indoor human location behaviour was studied by \todo{cite}Krajnik et al.  using Fourier transform methods and Gaussian mixture models. 

while Krajnik's  work is first step towards learning human mobility behaviour in indoor environments, it failed to address the problem of sparse dataset in domestic robots. Observations made by the robots are very sparse. We aim to demonstrate in this chapter that, by using Bayesian models, we can capture the human behaviour patterns in a sparse dataset.




\begin{thebibliography}{99}

\bibitem{c1} J.  McInerney,  J.  Zheng,  A.  Rogers,  and  N.  R.  Jennings.Modelling heterogeneous location habits in human populations for location prediction under data sparsity. In Interna-tional Joint Conference on Pervasive andUbiquitous Com-puting (UbiComp 2013)
\bibitem{c2}  S.  Scellato,   M.  Musolesi,   C.  Mascolo,   V.  Latora,   and A.  Campbell. Nextplace:   a  spatio-temporal  prediction framework for pervasive systems.  InPervasive Computing
\bibitem{c3} L. Song, D. Kotz, R. Jain, and X. He.  Evaluating next-cell predictors with extensive wi-fi mobility data. IEEE Trans-actions on Mobile Computing , 5(12):1633–1649, 2006 pages 152–169, San Francisco, CA, USA, 2011. Springer.
\bibitem{c4}N. Eagle and A. S. Pentland.   Eigenbehaviors:  identifying structure in routine. Behavioral Ecology and Sociobiology ,63(7):1057–1066, 2009.
\bibitem{c5} H.  Gao,  J.  Tang,  and  H.  Liu.   Exploring  social-historical ties on location-based social networks.  In6th InternationalAAAI Conference on Weblogs and Social Media, 2012
\bibitem{c6} Krajnik, Tomas, Miroslav Kulich, Lenka Mudrova, Rares Ambrus, and Tom Duckett. “Where’s Waldo at Time T? Using Spatio-Temporal Models for Mobile Robot Search.” In Robotics and Automation (ICRA), 2015 IEEE International Conference on, 2140–2146.

\end{thebibliography}



\section{Aruba Dataset}
In our thesis we used a  publicly-available  dataset  Aruba published by the Lincoln Center for Autonomous Systems, LCAS \todo{cite \cite{STRANDS}}. The dataset is of person presence collected at a smart apartment by the Center for Advanced Studies in Adaptive Systems, CASAS \todo{cite \cite{STRANDS}}.

The testbed where the dataset was collected is a  three-bedroom apartment located on the Washington State University that is part of CASAS smart home project. 
\begin{figure}[htp]
\centering
\includegraphics[width=\textwidth]{images/aruba-flat.png}
\caption{Aruba apartment visualization}
\label{aruba}
\end{figure}
As shown in Figure \cite{aruba}, the smart apartment test bed includes three bedrooms, one bathroom, a kitchen, and a living / dining room.  The apartment is equipped with motion sensors distributed approximately 1 meter apart throughout the space. The Aruba dataset was extracted from these motion sensor dataset provided by CASAS. The dataset contains the location of a person in the apartment every minute for 16 weeks.

\subsection*{Sparsification}
Aruba dataset is a large dataset as compared to an person location dataset we assume the robot will be able to generate. The Aruba dataset has recordings of every minute for 118 days, which is 161280 readings.
On the contrary the assumed dataset which will be collected by the robot by autonomously roaming in a home will be just 3-5 readings per day.
So for simulating the sparsity in the object location dataset we will sparsify the ARUBA dataset by random selecting only selecting 3-5 readings each day.

\subsection*{Visualization}
We visualize the dataset to find out if as humans we can find any patterns in the data. Since the observations are of 2 dimension(location vs time) the visualization is feasible. As explained in the section \todo{cite the central thesis}, we try to learn daily patterns by dividing the data into per hour cycle. As visualized in Figure \cite{aruba-visual} we can find that there are some prominent patterns emerging which can be learned. For example the usage of the Bedroom, Living room, outside and kitchen.

\begin{figure}[htp]
\centering
\includegraphics[width=\textwidth]{images/aruba-data.png}
\caption{Aruba Dataset Visualization : The X -axis are the locations of the home, Y-axis are the hours of the day. The intensity of the color in each box indicates the number of times the person is present in that location. Higher the intensity means more time is spent by the person in that location at that time.}
\label{aruba-visual}
\end{figure}


\section{Dirichlet Categorical model }

Dirichlet Categorical model (DCM) is a 2 level Bayesian model. The basic idea is that observations of each time period is characterized by a distribution over the possible locations.

The data are the observed human location $x_{ij}$ for $i = 1 \dots T$ time periods and then $j = 1, \dots , N$  are the observations.  We assume that the latent pattern in the persons location per period are distributed as a categorical distribution. The number of periods $T$ for our model is fixed to 24 corresponding to the number of hours in a day. 

The Dirichlet-Categorical model is the generalization of the Beta-Binomial model to multiple classes of a categorical or multinomial distribution. The conjugate prior for the categorical distribution is the Dirichlet distribution. The model estimates the posterior distribution of $\theta_i$ given our current data and prior beliefs. Our prior beliefs are encoded in the model through the hyperparameter $\alpha$, which represents pseudocounts of what we believe the data should look like – typically set as 1's for weak uniform beliefs. The graphical diagram of the model is shown in Figure \cite{dcm}

\noindent
\begin{figure}[htp]

\begin{minipage}{0.3\textwidth}
\centering

\tikz {
\node [const]                  (alpha) {$\alpha$};
\node [below=of alpha, latent]  (theta) {$\theta_i$};
\node [below=of theta, obs]     (x)     {$x_{ij}$};
\edge {alpha} {theta};
\edge {theta} {x};
\plate {trials} {(x)} {j data};
\plate {bags} {(theta)(x)(trials)} {i time};
}

\end{minipage}%
\begin{minipage}{0.7\textwidth}

\begin{equation*}
	\alpha = <1, 1, .... , 1 > 
\end{equation*}
\begin{equation*}
	\theta_i  \sim Dirichlet(\alpha)
\end{equation*}
\begin{equation*}
	x_{ij} \sim Categorical(\theta_i)
\end{equation*}
\end{minipage}
\caption{Graphical model representation of DCM. The boxes are ``plates" representing replicates. The outer plates represents hours of a day, while the inner plate represents the choice of places within each hour.}
\label{dcm}
\end{figure}



\section{Explanation : From LDA}
All the models in the thesis are based on the ``bag of words" assumption-- that the order of the data in a day can be neglected.
In language of probability theory, this is an assumption of \emph{exchangeability} for the words in the document (Aldous, 1985)

from : http://blog.nullspace.io/motivating-bayesian.html
an important class of models called exchangeable models are guaranteed by de Finetti’s theorem to decompose cleanly into the definition of the Bayesian framework.

One reason exchangeability is interesting is because prima facie, although it seems to be similar to the assumption of iid, it is actually much richer, and much less restrictive. Consider, for example, if we have a document whose words have been shuffled. Intuitively, if we see an Arabic word, then even it is much more likely that we will encounter other Arabic words elsewhere in the document, since there is a reasonable probability the document itself is in Arabic. The assumption of iid does not capture this, because the words are assumed to be completely independent. Exchangeability doesn’t assume independence, however, only that ordering doesn’t matter, so this will be captured in a model that is exchangeable.

This is a critical aspect of Bayesian inference that many people simply don’t understand, and it has all sorts of consequences, like making Bayesian nonparametrics possible, and making conditional probabilities uncomputable in general. It is so convenient to think of this class of problems as Bayesian that it seems like a crime to not use Bayesian inference.


\begin{itemize}
	\item Generative probabilistic model
	\item three-level hierarchical Bayesian model
	\item 
\end{itemize}


\section{Discussions}

http://andrewgelman.com/2016/08/22/bayesian-inference-completely-solves-the-multiple-comparisons-problem/
http://andrewgelman.com/2013/11/21/hidden-dangers-noninformative-priors/

The point of the story in that slide is that flat priors consistently give bad inferences. Or, to put it another way, the routine use of flat priors results in poor frequency properties in realistic settings where studies are noisy and effect sizes are small.
\label{sec:}

% section  (end)
\end{document}



%######################################################################

\chapter{Knowledge-enabled fault tolerant object search}
\label{cha}

Efficient searching for objects in the environment is one of the application for the knowledge being learned from object locations. The learned knowledge is as heuristics to improve the search time of objects. There are 2 basic methods in which we can use the learned knowledge as heuristics, first we can search for objects in the decreasing order of the learned probabilities or alternatively we can use the descriptive model to predict based on its learned probabilities. However the probabilistic representation of the learned knowledge can be used in ingenious way to solve more complex problems. Markov Decision Process (MDP) based planners can use the learned probabilities in their decision process to make more informed decisions. To illustrate this flexibility we have used the learned probabilities to make a sequential decision search algorithm which can accommodate the recognition failure in the vision algorithms.
\missingfigure{Image with false detection}
Household robots need to be able to recognize the objects they’re supposed to search and manipulate. But while object recognition is one of the most widely studied topics in artificial intelligence, even the best object detectors still fail much of the time. As a result there are very much possibilities that even though the search algorithm has provided the correct location the object detector might fail to locate the object. An alternative will be provide a search algorithm which can accommodate the object detector failures. All these object detectors along with the detection also publish the confidence(probability) of the detection. The proposed algorithm here combines both the \emph{detection probability} and the \emph{learned probability} of the objects and provide sequential decisions for the next location to search.

\section{Search as a decision Framework}
\todo[inline, caption={search1}]{Introduce the topic name}
\todo[inline, caption={search2}]{Add the 2 papers as the related work of probabilistic search in robotics}
\todo[inline, caption={search3}]{ and that the below model is a reduced version and simplified model}

The  contributions  of  this  paper  include  the  formulation of the search control problem as a decision-making problem rather  than  a  sensing  task,  where  measures  associated  with decisions,  e.g.  confidence  and  robustness  of  the  decision or  time  until  the  decision  is  made,  are  used  to  design  an appropriate  control  policy.  Our  formulation  of  the  search problem  as  a  detection  problem  also  allows  us  to  include practical  sensor  artifacts  (such  as  false  alarms  and  missed detections)  which  have  not  be  completely  considered  in other  formulations  [6]  of  the  search  problem.


A Decision-Making Framework for Control Strategies
in Probabilistic Search
Timothy H. Chung and Joel W. Burdick

A probabilistic framework for object
search with 6-DOF pose estimation 
\todo[inline]{robotics search in robots as POMDP in its related work}
\section{Model}
The idea is use the learned knowledge of the object locations as the first guess (prior ) of the probabilities that the object in question is located in each of the possible object locations. The prior probabilities suggest which location to search first. If the object is not found in that location, the prior probabilities are then updated (yielding the posterior), and the process is repeated until the object is found. 

Assume that the domain of interest is a home we call $D_s$, which is made up of n spatial areas. Let $Y_i = 1$ if the object is in the \emph{i}th location, and then $Y_i = 0$ if it is not; $ i = 1, .... , n$. Now as discussed above object detectors can fail to detect the object. So, let $Z_i = 1$ if the object is found in the \emph{i}th location, and $Z_i = 0$ if not.

We can define two terms \emph{detection probability},
\begin{equation}
	p_i = Pr(Z_i = 1| Y_i =1), \qquad  i = 1,...,n,
\end{equation}
which is a conditional probability, and the \emph{occurrence probability},
\begin{equation}
	\pi_i = Pr(Y_i = 1),\qquad  i = 1, .... , n.
\end{equation}

This can be expressed in probabilistic programming form:
\begin{gather}
	Z_i | Y_i \sim Bernoulli (Y_i, p_i), \qquad  i  = 1,....,n \\
	Y_i \sim Bernoulli(\pi_i),\qquad   i = 1,...,n,
\end{gather}

where, the occurrence probability {$\pi_i$} is given by the learned probabilities, while the detection probability {$\pi_i$} is provided by the object detector algorithm.

Now, assume that hte \emph{i}th location is searched by the robot and the object is not found (i.e., $Z_i = 0$). In that case, the probability that the object is in the \emph{i}th location is updated using Bayes' Theorem. This yields the posterior probability,

\begin{align*}
	Pr(Y_i = 1 | Z_i = 0) &= \frac{Pr(Z_i = 0|Y_i=1)Pr(Y_i=1)} {Pr(Z_i=0)} \\
	                       &= \frac{Pr(Z_i = 0| Y_i =1 )Pr(Y_i =1)}{Pr(Z_i=0|Y_i =1)Pr(Y_i = 1) + Pr(Z_i=0|Y_i=0)Pr(Y_i=0)} \\
	                       &= \frac{(1 - p_i)\pi_i}{(1 - p_i)\pi_i + (1)(1 - \pi_i)} \\
	                       &= \frac{(1 - p_i)\pi_i}{1 - p_i\pi_i}
\end{align*}

where we assume that there are no false-positive detection (i.e., $Pr(Z_i =0| Y_i = 0) = 1$). Note that the new posterior is less than the prior probability $\pi_i$ as the object was not observed.

Since the object is not found in the searched location, then this should also affect the posterior probability in the other grid boxes. For example, consider the \emph{j}th location, where $j \neq i$. Then,
\begin{align*}
	Pr(Y_j = 1 | Z_i=0) &= \frac{Pr(Z_i = 0 | Y_j = 1)Pr(Y_j = 1)}{Pr(Z_i = 0)} \\
	                    &= \frac{\pi_j}{ 1 - p_i\pi_i}
\end{align*}

Thus, the posterior probability of the object being the \emph{j}th location is greater than the prior probability $\pi_j$ . These new posteriors will then become the next prior probabilities is a sequential procedure that would determine the next location to search.

\section{Example}
\todo{split as 2 example , 1 with high detection probablity 0.9 and other with 0.6}
\todo{show the different sequence}
Suppose we have a domestic robot which has learned the location probabilities of a cup in a home. The robot has learned that the cup can be found 3 locations in the home, thus the domain for search can be defined as $D_s = \{cupboard, table, dishwasher \}$ . 
Now lets assume the robot has been in the room for long enough and made some observations. Based on these observations it learns the occurrence probabilities as $p_i = \{ 0.8, 0.1, 0.1\}$ . The object detector for the cup has a \emph{detection probability} of $p_{cup} =  0.6$ .

As you can see there is a strong probability of finding the cup in the cupboard. The algorithm selects to search the cupboard. But the object detector fails and is not able to detect the cup. If we are not using the algorithm the robot will move on to the next location to search the cup in the next location.


\missingfigure{Illustration of the proabbiliities and how they change with each observation}

However our algorithm on the failure detection will update its probabilities and the updated posterior probability will be $ p_i = \{ 0.44, 0.27, 0.27\}$ and it still selects to search the cupboard for the next time. This illustrates the algorithm understands the limitation in the object detector and updates its beliefs to accommodate this limitation and make better \emph{informed} decisions.  

 
\section{Evaluation}
hey you didnt think about evaluation ? 
% chapter  (end)

\chapter{Cleaning robot scenario}

If you grew up watching the Jetsons, the idea of having a robot helper that cleans and cooks for your family has always been a fantasy. People want to have a robot do the housework and the good news is that great minds are working on it.

Elon Musk announced back in June that his OpenAI robotic institute was working on developing software that would enable off-the-shelf robots to do household chores and various other engineers are working on robots that could accomplish cleaning and organizing tasks.

 In order for a robot to clean an entire house, it would have to be able to fully understand its environment and make decisions based on the state of each room -- something that is beyond the current abilities of artificial intelligence. 
 \cite{http://www.treehugger.com/gadgets/want-robot-clean-your-house-youll-have-wait-bit-longer.html}
 
\chapter{Modeling}
Data Modeling.
Each data scientist in practice uses tools and viewpoints from
both
of
Leo Breiman's modeling cultures:

Generative modeling
, in which one proposes a stochastic model that could have generated
the data, and derives methods to infer properties of the underlying generative mechanism.
This roughly speaking coincides with traditional Academic statistics and its o shoots.
31

Predictive modeling
, in which one constructs methods which predict well over some some
given data universe { i.e.  some very speci c concrete dataset.  This roughly coincides with
modern Machine Learning, and its industrial o shoots.

\chapter{Useful wordings}

\begin{itemize}
	\item Experience based machine intelligence \textbf{and the benefits it could engender such as} fixing model-specific problems and product efficiency \textbf{are good reasons to enable network connectivity}.
	\item Not all data is created equal, and certainly not all of it is meaningful to collect and display.
\end{itemize}

%\listoftodos
\end{document}
