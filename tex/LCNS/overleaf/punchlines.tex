% This is based on the LLNCS.DEM the demonstration file of
% the LaTeX macro package from Springer-Verlag
% for Lecture Notes in Computer Science,
% version 2.4 for LaTeX2e as of 16. April 2010
%
% See http://www.springer.com/computer/lncs/lncs+authors?SGWID=0-40209-0-0-0
% for the full guidelines.
%
\documentclass{llncs}
\usepackage{graphicx}
\begin{document}

\title{Learning Probabilistic Models for Object Location }
%
\titlerunning{Hamiltonian Mechanics}  % abbreviated title (for running head)
%                                     also used for the TOC unless
%                                     \toctitle is used
%
\author{Deebul Nair\inst{1} \and Tim Niemueller \inst{2}
\and Gerhard Pl\"{o}ger \inst{1} \and Gerhard Lakemeyer \inst{2}}
%
\authorrunning{Deebul Nair et al.} % abbreviated author list (for running head)
%
%
\institute{Bonn-Rhein-Sieg University of Applied Sciences\\Department of Computer Science\\ 
Sankt Augustin, Germany\\
\email{{<deebul.nair, paul.ploeger>}@inf.h-brs.de},
\and
Knowledge-based Systems Group\\
RWTH Aachen University, Aachen, Germany \\
\email{{<niemueller, gerhard>}@kbsg.rwth-aachen.de}}
\maketitle              % typeset the title of the contribution

\begin{abstract}
In this paper our aim is to predict locations of objects in a particular home environment, where the robot has the map of the environment and the knowledge of previous locations of the objects . We argue that by making use of temporal information for prediction improves the prediction of the location of an object. To realize this, we develop a probabilistic models to capture the dynamics of observed objects, called as the object location model. The object location model will make use of both spatial and temporal information recorded by the robot to make predictions. Even though all homes show much similarity in their spatial organization, the usage of the space and the objects in it is entirely based on the user’s preferences. Most of the previous work on object location prediction is based upon predicting the object’s location in a generic home while the proposed model will make predictions for a specific home. We demonstrate the validity of the approach by comparing it with the state of the art prediction algorithm. 
\keywords{Probabilistic Models, Spatio-Temporal learning}
\end{abstract}
%
\section{Publishable results}
\begin{itemize}
	\item Probabilistic models perform better than the classical machine learning algorithms(SVM and Random forests)
	\item Temporal information improves the prediction.
	\item Search time for objects improved than the state of the art models.
\end{itemize}

\section{Punchlines}
\begin{itemize}
	\item Even though all homes show much similarity in their spatial organization, the usage of the space and the objects in it is entirely based on the user’s preferences. Most of the previous work on object location prediction is based upon predicting the object’s location in a generic home while the proposed model will make predictions for a specific home.
	\item The novelty of our Bayesian model lies in the ability to draw information from the network data as well as from the associated categorical outcome in a unified hierarchical model for classification. In addition, our method allows for intuitive integration of a priori network information directly in the model and allows for posterior inference on the network topologies both within and between classes.
	\item The dataset of object locations used for prediction is generated by the robot running in a domestic environment. The dataset is very sparse because of occlusions and limitation in the object recognition algorithms. We have developed approaches which learn the object locations based on the small dataset.
\end{itemize}
\end{document}
